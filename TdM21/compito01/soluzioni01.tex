\documentclass[10pt]{article}
\usepackage[utf8]{inputenc}
\usepackage[a4paper,height=24cm,width=13cm]{geometry}
\usepackage[italian]{babel}
\usepackage{amssymb}
\usepackage{dsfont}
\usepackage{calc}
\usepackage{graphicx}
\usepackage{pstricks}
\usepackage{pst-node}
\usepackage{fourier}
\usepackage{euscript}
\usepackage{amsmath,amssymb, amsthm}

\def\lh{\textrm{lh}}
\def\phi{\varphi}
\def\P{\EuScript P}
\def\M{\EuScript M}
\def\D{\EuScript D}
\def\U{\EuScript U}
\def\S{\EuScript S}
\def\sm{\smallsetminus}
\def\niff{\nleftrightarrow}
\def\ZZ{\mathds Z}
\def\NN{\mathds N}
\def\PP{\mathds P}
\def\QQ{\mathds Q}
\def\RR{\mathds R}
\def\<{\langle}
\def\>{\rangle}
\def\E{\exists}
\def\A{\forall}
\def\0{\varnothing}
\def\imp{\rightarrow}
\def\iff{\leftrightarrow}
\def\IMP{\Rightarrow}
\def\IFF{\Leftrightarrow}
\def\range{\textrm{im}}
\def\Mod{\textrm{Mod}}
\def\Aut{\textrm{Aut}}
\def\Th{\textrm{Th}}
\def\acl{\textrm{acl}}
\def\eq{{\rm eq}}
\def\tp{\textrm{tp}}
\def\equivL{\stackrel{\smash{\scalebox{.5}{\rm L}}}{\equiv}}
\def\swedge{\mathbin{\raisebox{.2ex}{\tiny$\mathbin\wedge$}}}
\def\svee{\mathbin{\raisebox{.2ex}{\tiny$\mathbin\vee$}}}

\newcommand{\labella}[1]{{\sf\footnotesize #1}\hfill}
\renewenvironment{itemize}
  {\begin{list}{$\triangleright$}{%
   \setlength{\parskip}{0mm}
   \setlength{\topsep}{0mm}
   \setlength{\rightmargin}{0mm}
   \setlength{\listparindent}{0mm}
   \setlength{\itemindent}{0mm}
   \setlength{\labelwidth}{3ex}
   \setlength{\itemsep}{0mm}
   \setlength{\parsep}{0mm}
   \setlength{\partopsep}{0mm}
   \setlength{\labelsep}{1ex}
   \setlength{\leftmargin}{\labelwidth+\labelsep}
   \let\makelabel\labella}}{%
   \end{list}}
%\def\ssf#1{\textsf{\small #1}}
\newcounter{ex}
\newenvironment{exercise}{\clearpage\addtocounter{ex}{1}\textbf{Esercizio \theex.\quad}}{}
\newcounter{sol}
\newenvironment{solution}{\addtocounter{sol}{1}\textbf{Soluzione \theex.\quad}}{}
\pagestyle{empty}
\parindent0ex
\parskip2ex
\raggedbottom
\def\nsR{{}^*\!\RR}
\def\ssf#1{\textsf{#1}}
\renewcommand{\baselinestretch}{1.3}


\usepackage{fancyhdr}
\pagestyle{fancy}
\lfoot{Teoria dei Modelli a.a.~2020/21}
\rhead{}
\rfoot{\rput(1.5,-0.5){\small\thepage}}
\cfoot{}







%
% INDICARE NOME E COGNOME DI TUTTI GLI AUTORI
%
\lhead{Gabriele Rastello\quad +\quad Mattia Viscariello}

%
\begin{document}

\begin{exercise}
  Let $M$ be an $L$-structure and let $\psi(x), \phi(x,y)\in L$. For each of the following conditions, write a sentence true in $M$ exactly when
  \begin{itemize}
  \item[a.] $\psi(M)\ \in\ \big\{\phi(a,M): a\in M\big\}$;
  \item[b.] $\big\{\phi(a,M): a\in M\big\}$ contains at least two sets;
  \item[c.] $\big\{\phi(a,M): a\in M\big\}$ contains only sets that are pairwise disjoint.
  \end{itemize}
\end{exercise}

\begin{solution}
  \begin{itemize}
  \item[a.] \(\exists x\forall y (\psi(y)\leftrightarrow\phi(x, y))\);
  \item[b.] \(\exists x, y, z (\phi(x, y) \nleftrightarrow \phi(x, z))\);
  \item[c.] \(\forall x, y (x \neq y \rightarrow\exists z(\phi(x, z) \nleftrightarrow \phi (x, z)))\)
  \end{itemize}
\end{solution}

\begin{exercise}
  Let $M$ be a structure in a signature that contains a symbol $r$ for a binary relation.
  Write a sentence $\phi$ such that
  \begin{itemize}
  \item[a.] $M\models\phi$ if and only if there is an $A\subseteq M$ such that $r^M\ \subseteq\ A\times\neg A$.
  \end{itemize}
\end{exercise}

\begin{solution}
  Let \(\phi\) be the formula \(\forall x\forall y (xry \rightarrow\forall z \neg(yrz))\).

  If \(M\models\phi\) then set \(A = \big\{a\in M\colon \text{there is a \(b\in M\) such that \(ar^Mb\) }\big\}\).
  Now if \((a, b)\in r^M\) then \(a\in A\) by definition of \(A\) and there is no \(c\in M\) such that \(b r^M c\) (because \(M\models\phi\)) so \(b\not\in A\).
  This proves that \(r^M\subseteq A\times\neg A\).

  Conversely suppose that \(r^M\subseteq A\times\neg A\) for some \(A\subseteq M\).
  If \(a r^M b\) we immediately have \(b\in\neg A\).
  Now for the sake of contradiction let there be \(c\in M\) such that \(b r^M c\); but this immediately implies \(b\in A\) that is absurd.
  We are forced to conclude that \(M\models\phi\).
\end{solution}

\begin{exercise}
  Let $M\preceq N$ and let $\phi(x)\in L(M)$.
  Prove that $\phi(M)$ is finite if and only if $\phi(N)$ is finite and in this case $\phi(N)=\phi(M)$.
\end{exercise}

\begin{solution}
  We recall that \(M\preceq N\) means that \(M\) is a \(L(M)\)-substructure of \(N\) such that  \(N\models\psi\) \textit{if and only if} \(M\models \psi\) for all sentences \(\psi\in L(M)\).
  We thus trivially have \(\phi(M)\subseteq\phi(N)\) so \(\phi(N)\) finite implies \(\phi(M)\) finite.

  Now suppose \(\phi(M) = \big\{m_1,\ldots, m_k\big\}\) and \(\phi(M)\subset\phi(N)\).
  We thus have that \(N\models\psi\) where \(\psi\) is the formula
  \[\exists x(x\neq m_1 \land\ldots\land x\neq m_k\land \phi(x)).\]
  But \(\psi\) is a \(L(M)\)-sentence and thus \(M\models\psi\).
  This is clearly impossible because there sould be an element \(m\in M\) such that \(M\models\phi(m)\) but \(m\not\in\phi(M) = \big\{m_1,\ldots,m_k\big\}\).
  By contradiction we have \(\phi(N)\subseteq\phi(M)\).

  We conclude that \(\phi(M)\) finite implies \(\phi(M) = \phi(N)\) and thus \(\phi(N)\) finite as well.
\end{solution}

\begin{exercise}
  Let $M\preceq N$ and let $\phi(x,z)\in L$.
  Suppose there are finitely many sets of the form $\phi(a,N)$ for some $a\in N^{|x|}$.
  Prove that all these sets are definable over $M$.
\end{exercise}

\begin{solution}
  We present a solution for the case \(|x| = 1\) that we believe should be adjustable to the case \(|x| > 1\).
  We know that \(A = \big\{\phi(a, N)\colon a\in N\big\}\) is finite.
  For the sake of the argument we define the following equivalence relation on the elements of \(N\)
  \[a\sim b\quad\Leftrightarrow\quad \phi(a, N) = \phi(b, N).\]
  One immediately has that if \(a\sim m\) for some \(m\in M\) then \(\phi(a, N)\) is definable over \(M\) (just consider the formula \(\phi(m, x)\in L(M)\)).
  We will show that for all \(a\in N\) there is some \(m\in M\) such that \(a\sim m\) and thus that all sets in \(A\) are definable over \(M\).

  Choose \(m_1,\ldots, m_k\) in \(M\) such that if \(m\in M\) then \(m\sim m_i\) for some \(1\leq i\leq k\); we can always chose a finite number of such elements since \(A\) finite implies that \(\sim\) has a finite number of equivalence classes.
  Now assume that there is \(n\in N\) such that \(n\not\sim m_i\) for all \(1\leq i\leq k\).
  This condition is equivalent to requiring that \(N\) satisfies the \(L(M)\)-sentence
  \[\psi \equiv \exists x \bigwedge_{i=1}^k\neg(x\sim m_i).\footnote{We use \(x\sim y\) as an abbreviation for \(\forall z (\phi(x, z) \leftrightarrow \phi(y, z))\).}\]
  But now since \(M\preceq N\) we have \(M\models\psi\) which is absurd by the choice of \(m_1,\ldots,m_k\).
  By contradiction we conclude that for all \(a\in N\) there is some \(m\in M\) such that \(a\sim m\) and thus \(\phi(a, N)\) is definable over \(M\).
\end{solution}
\end{document}
