\documentclass[10pt]{article}
\usepackage[utf8]{inputenc}
\usepackage[a4paper,height=24cm,width=13cm]{geometry}
\usepackage[italian]{babel}
\usepackage{amssymb}
\usepackage{dsfont}
\usepackage{calc}
\usepackage{graphicx}
\usepackage{pstricks}
\usepackage{pst-node}
\usepackage{fourier}
\usepackage{euscript}
\usepackage{amsmath,amssymb, amsthm}

\def\lh{\textrm{lh}}
\def\phi{\varphi}
\def\P{\EuScript P}
\def\M{\EuScript M}
\def\D{\EuScript D}
\def\U{\EuScript U}
\def\S{\EuScript S}
\def\sm{\smallsetminus}
\def\niff{\nleftrightarrow}
\def\ZZ{\mathds Z}
\def\NN{\mathds N}
\def\PP{\mathds P}
\def\QQ{\mathds Q}
\def\RR{\mathds R}
\def\<{\langle}
\def\>{\rangle}
\def\E{\exists}
\def\A{\forall}
\def\0{\varnothing}
\def\imp{\rightarrow}
\def\iff{\leftrightarrow}
\def\IMP{\Rightarrow}
\def\IFF{\Leftrightarrow}
\def\range{\textrm{im}}
\def\Mod{\textrm{Mod}}
\def\Aut{\textrm{Aut}}
\def\Th{\textrm{Th}}
\def\acl{\textrm{acl}}
\def\eq{{\rm eq}}
\def\tp{\textrm{tp}}
\def\equivL{\stackrel{\smash{\scalebox{.5}{\rm L}}}{\equiv}}
\def\swedge{\mathbin{\raisebox{.2ex}{\tiny$\mathbin\wedge$}}}
\def\svee{\mathbin{\raisebox{.2ex}{\tiny$\mathbin\vee$}}}

\newcommand{\labella}[1]{{\sf\footnotesize #1}\hfill}
\renewenvironment{itemize}
  {\begin{list}{$\triangleright$}{%
   \setlength{\parskip}{0mm}
   \setlength{\topsep}{0mm}
   \setlength{\rightmargin}{0mm}
   \setlength{\listparindent}{0mm}
   \setlength{\itemindent}{0mm}
   \setlength{\labelwidth}{3ex}
   \setlength{\itemsep}{0mm}
   \setlength{\parsep}{0mm}
   \setlength{\partopsep}{0mm}
   \setlength{\labelsep}{1ex}
   \setlength{\leftmargin}{\labelwidth+\labelsep}
   \let\makelabel\labella}}{%
   \end{list}}
%\def\ssf#1{\textsf{\small #1}}
\newcounter{ex}
\newenvironment{exercise}{\clearpage\addtocounter{ex}{1}\textbf{Esercizio \theex.\quad}}{}
\newcounter{sol}
\newenvironment{solution}{\addtocounter{sol}{1}\textbf{Soluzione \theex.\quad}}{}
\pagestyle{empty}
\parindent0ex
\parskip2ex
\raggedbottom
\def\nsR{{}^*\kern-.2ex\RR}
\def\nsN{{}^*\kern-.2ex\NN}
\def\nsQ{{}^*\kern-.2ex\QQ}
\def\ns{{}^*\kern-.2ex}
\def\st{{\rm st}}
\def\ssf#1{\textsf{#1}}
\renewcommand{\baselinestretch}{1.3}




\usepackage{fancyhdr}
\pagestyle{fancy}
\lfoot{Teoria dei Modelli a.a.~2020/21}
\rhead{}
\rfoot{\rput(1.5,-0.5){\small\thepage}}
\cfoot{}

%
% INDICARE NOME E COGNOME DI TUTTI GLI AUTORI
%
\lhead{Nome/i Cognome/i\quad +\quad Nome/i Cognome/i}
%
\begin{document}

\begin{exercise}
  Assume $L$ is countable and let $M\preceq N$ have arbitrary (large) cardinality.
  Let $A\subseteq N$ be countable.
  Prove there is a countable model $K$ such that $A\subseteq K\preceq N$ and $K\cap M\preceq N$ (in particular, $K\cap M$ is a model).
  Hint: adapt the \textit{construction\/} used to prove the downward L\"owenheim-Skolem Theorem.
\end{exercise}

\begin{solution}
  We have \(|L(A)| = \aleph_0\).
  We shall construct a chain \(\big(A_i\big)_{i<\omega}\) of subsets of \(N\) such that \(K = \bigcup_{i<\omega}A_i\) will be the wanted model.

  We begin by setting \(A_0 = A\) then we have \(|A_0| = \aleph_0\) so \(|L(A_0)| = \aleph_0\).
  Now let \(\big(\phi_k(x)\colon k < \aleph_0\big)\) be an enumeration of all the formulas of \(L(A_0)\) that are consistent in \(N\).
  By construction there is a sequence \(\big(a_k\in N\colon k < \aleph_0\big)\) such that \(N\models \phi_k(a_k)\) for all \(k<\aleph_0\);
  moreover if \(\phi_k(x)\in L(M)\) we can chose \(a_k\) in \(M\) because of the Vaught-Tarski test.
  Finally we set \(A_1 = A_0 \cup \big\{a_k\in N\colon k<\aleph_0\big\}\); note that \(|A_1|=\aleph_0\).
  By iteration of this process we obtain the chain \(\big(A_i\big)_{i<\omega}\).

  As in the proof of the Downward L\"owenheim-Skolem by construction we have that \(|K| = \aleph_0\) and the Vaught-Tarski test can be used to show that \(K\preceq N\).

  Similarly if \(\phi(x)\in L(K\cap M)\) then \(\phi(x)\in L(M)\) and so \(K\) will contain a solution that is also in \(M\) thus \(K\cap M\preceq N\) by the Vaught-Tarski test (and particularly \(K\cap M\) is a model).
\end{solution}
  
\begin{exercise}
  Prove either that \ssf{1a}$\IFF$\ssf{2a}, or that \ssf{1b}$\IFF$\ssf{2b}. (Choose whichever you prefer.)

  \begin{minipage}{.5\textwidth}
    \begin{itemize}
    \item[1a.] $X\subseteq\RR$ is open;
    \item[2a.] $b\approx a\in X \ \IMP\ b\in\ns X$ for every $b\in\nsR$.
    \end{itemize}
  \end{minipage}
  \begin{minipage}{.5\textwidth}
    \begin{itemize}
    \item[1b.] $X\subseteq\RR$ is closed;
    \item[2b.] $a\in\ns X \ \IMP\ \st\, a\in X$ for every finite $a\in\nsR$.
    \end{itemize}
  \end{minipage}
  
  Open and closed are understood w.r.t.\@ the usual tolopology on $\RR$.
\end{exercise}

\begin{solution}
  We prove that \ssf{1a}\(\IFF\)\ssf{2a}.

  Assume \(X\) open and let \(a\) be one of its points; then we know that there is some real \(r>0\) such that \(\RR\models\forall x \big(|a - x| < r \imp x \in X\big)\).
  Since it is an elementary extension we have that this sentence is also true in \(\nsR\).
  Now let \(b\in\nsR\) be such that \(b\approx a\); then \(|b - a|\) is an infinitesimal and as such \(|b - a| < r\) so \(b\in \ns X\).

  Now fix \(a\in X\) and assume \ssf{2a}.
  Note that if \(\delta>0\) is an infinitesimal then \(|a-x| < \delta\) implies \(a\approx x\) and thus \(x\in \ns X\).
  Since non-zero infinitesimals exist (there are actually infinitely many) \(\nsR\models\exists\delta\big(\delta>0 \land \forall x\big(|a - x|\leq\delta \imp x\in\ns X\big)\big)\).
  But now \(\RR\) is an elementary substructure of \(\nsR\) so \(\RR\models\exists\delta\big(\delta>0 \land \forall x\big(|a-x| < \delta \imp x\in X\big)\big)\).
  Finally, since the choice of \(a\) is arbitrary, \(X\) is an open set.
\end{solution}

\begin{exercise}\kern-1ex\textbf{(Bonus question)}\kern1ex
  For which sets $X \subseteq \RR$ does the following hold?
  \begin{itemize}
  \item[2.] $b\approx a\in\ns X \ \IMP\ b\in\ns X$ for every $a,b\in\nsR$.
  \end{itemize}
\end{exercise}
\end{document}
